\documentclass{article}
\usepackage{graphicx}
\usepackage[utf8]{inputenc}
\usepackage[english]{babel}
\usepackage[document]{ragged2e}

\begin{document}

\title{Assigment 1}
\author{Yogesh Bhosale}

\maketitle

\begin{itemize}


\item Question 1
\begin{itemize}
\item Any number in range $\{0,...,2^{k}-1\}$ can be represented as unique sequence of $k$
bits in its binary form.
\item If we call $rangebit$ $k$ times it will generate a number in range $\{0,...,2^{k}-1\}$.
\item Each number would occur with probabilty $p$,
\begin{flushleft}
\begin{equation}
    p(x) = p_1*p_2*...*p_k
\end{equation}
\begin{equation}
    p(x) = \frac{1}{2} * \frac{1}{2} *...*\frac{1}{2}
\end{equation}
\begin{equation}
    p(x) = \frac{1}{2^{k}}
\end{equation}
\end{flushleft}
\item If $n$ is a power of $2$ then $n=2^{ \lg n }$, in this case Equation 3 becomes,
   \begin{equation}
    p(x) = \frac{1}{2^{\lg n}}
\end{equation}
\begin{equation}
    p(x) = \frac{1}{n}
\end{equation}
\item If $n$ is not a power of $2$ then we can generate $\left \lceil \lg n \right \rceil$ random bits. This will generate a random number in a range $\{0,...,2^{\left \lceil \lg n \right \rceil}\}$.
\end{itemize}



\item Question 2
\begin{itemize}
\item Longest streak length : 9 \linebreak
\item Source file is attached with name $Q2.java$
\end{itemize}



\item Question 4
\begin{itemize}
\item Size of a mincut : 24 \linebreak
\item Number of iterations needed to detect mincut : 27 \linebreak
\item Source file is attached with name $Q4.java$
\end{itemize}


\end{itemize}

%\begin{equation}
%    \label{simple_equation}
%    \alpha = \sqrt{ \beta }
%\end{equation}

\end{document}